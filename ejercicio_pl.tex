\documentclass{article}
\usepackage{amsmath}
\usepackage[utf8]{inputenc}
\usepackage{amsfonts}

\begin{document}
Considere el problema:
\begin{equation*}
  \begin{aligned}
    \text{Maximizar} &\quad x+y \\
    \text{sujeto a} & \quad
    \begin{aligned}
      x & \geq 0\\
      y & \geq 0\\
      2x+y & \leq 2\\
    \end{aligned}
  \end{aligned}
\end{equation*}
\begin{itemize}
\item Resulve el problemapor el métodografico.
\item Escribe el problema en su forma estándar, determinando A, b y c.
\item Dibuja la región factible del problema estándar de $\mathbb{R}^{3}.$
\item Determina las soluciones factibles básicas del problema estándar.
\item Evalúa la función objetivo en las soluciones factibles básicas para determinar la solución óptima. Confirma el resultado anterior.
  \item 
\end{itemize}
Considere el problema
\begin{equation*}
  \begin{aligned}
    \text{Maximizar } & x_1+x_2\\
    \text{sujeto a } &
    \begin{aligned}
      0 &\leq x_1\\
      0 &\leq x_2\leq 1\\
      2 &\geq 2x_1+x_2
    \end{aligned}
  \end{aligned}
\end{equation*}
\begin{itemize}
\item Resuelve el problema por el método gráfico.
\item Escribe el problema en su forma estándar, determinando A,b,c.
\item Determina todas las soluciones factibles básicas del problema estándar.
\item Evalúa la función objetivo en las soluciones factibles básicas para determinar la solución óptima. Confirma el resultado anterior.
\end{itemize}
Considerar el siguiente problema:
\begin{equation*}
  \begin{aligned}
    \text{minimizar}&\quad x+y\\
    \text{sujeto a}&\quad
    \begin{aligned}
      x&\geq0\\
      y&\geq0\\
      2x+y&\geq2\\      
    \end{aligned}
  \end{aligned}
\end{equation*}
\begin{itemize}
\item Resuelve el problema por el método gráfico.
\item Escribe el problema en su forma estandar, determina.
\item Dibuja la región factible del problema estandar en $\mathbb{R}^{3}$.
\item Determina las soluciones básicas del problema.
  \item Evalua la solución objetivo en las soluciones básicas para determinar la solución óptima. Confirma el resultado anterior.
\end{itemize}

\end{document}
